\documentclass[8pt]{report}
\usepackage{multicol}
\usepackage[a4paper, portrait, margin=0.5in]{geometry}
\setlength\parindent{0pt}
\setlength{\abovedisplayshortskip}{-1pt}
\setlength{\belowdisplayshortskip}{0pt} 
\usepackage{graphicx}
\usepackage{mathtools}
\pagenumbering{gobble}

\begin{document}

Microeconomics is social science and business tool. Creating economic value alone though is insufficient, it is necessary to \textit{capture} that value.

\begin{multicols}{3}
[
\section{Determine Prices}
In competitive markets
]
Principles
\begin{enumerate}
\item no free lunch, we face trade-offs
\item opportunity cost is what matters
\item rational actors think at the \textit{margin}
\item individuals react to \textit{incentives}
\end{enumerate}

Perfectly competitive market 
\begin{itemize}
\item firms make selling decisions, taking the market price as given = market supply curve 
\item buyers make purchase decision taking the market price as iven = market demand curve
\item equilibrium market price is reached $Q_{demanded} = Q_{supplied}$
\end{itemize}

Costs: 
\begin{itemize}
\item fixed cost: cost that cannot be avoided in the short run
\item variable cost: cost that can be avoided; they may vary continuously with output
\item total cost: $\sum FC + VC$
\item full-reinvestment total cost: $FC + VC + Annual_{CapitalCharge}$ \\ fixed cost plus variable costs plus the annual capital charge: which includes the yearly financing cost of building the company
\item marginal cost: extra cost of producing one extra unit 
\end{itemize}

Features of Marginal Cost Curve:
$$
	MC = RATE[TC \; and \; TVC \delta]
$$

$$
	MC = \frac{\partial TC}{\partial Q} = \frac{\partial TVC}{\partial Q}
$$
Therefore, if TC is linear, MC is a horizontal line corresponding to slope of TC.

\end{multicols}

\hrule

\begin{multicols}{3}

for TVC is linear: \\ $MC = AVC == AVC = \frac{MC \cdot Q}{Q}$ \\
Therefore, supply decision of an individual is either ZERO or FULL Capacity \\
This does not suggest breakeven: firms may operate at below ATC in the short-run \\

When ATC and FR-ATC are U-Shaped: \\
MC becomes a linear line (derivative of a parabolic function) \\
\textbf{Short run equilibrium price is the intersection of MC with FR-ATC} \\

Conditions for LR Equilibirum: \\
\begin{enumerate}
	\item no incentive for new entry: so $P = MIN[FR-ATC]$
	\item profit maximization by firms, each firm produces Q
	\item supply = demand
\end{enumerate}

Demand collapse: \\
New equilib price falls below ATC. This forces firms to withdraw capacity until it reaches ATC. In the long run, if demand stays depressed, market forces will eliminate capacity from the industry when prices are stuck between $min[ATC]$ and $min[FR-ATC]$

\end{multicols}

\hrule

\begin{multicols}{3}

consumer suprlus = area between market price and demand curve \\
producer surplus = area between market price and supply curve \\
$$
	Economic_{value} = Surplus_{C} + Surplus_{P}
$$

\end{multicols}

\begin{multicols}{3}
[
\section{Government}
Interventions
]
The economics of taxes: \\
A tax introduction shifts the supply curve leftward, cause a rise in price. However, this price lift is shared between the consumer and the producer - resulting in a dead-weight-loss of D and F. Session 5 - page 9. \\

Subsidies: \\
The US Cotton Subsidy shifted the entire USA supply to the left of the supply curve - thus lowering the entire equilibirum price of global cototn prices \\

Price Elasticity:
$$
	\epsilon = \frac{ \frac{\partial Q_D}{\partial P}}{P/Q}
$$

US Subidies implications and thoughts:
\begin{enumerate}
\item tax cost to US taxpayers is far more than extra profit enjoyed by US producers 
\item elimination of subidies would increase earnings from growing cotton outside the US: the US foreign aid budget was much higher than this
\item subidies reduce total world surplus
	\begin{enumerate}
		\item higher overall costs (crowding out in the market)
		\item gain from extra consumption is not worth it! 
	\end{enumerate}
\end{enumerate}

So why have subidies: \\
\begin{enumerate}
\item dispersed costs
\item concetrated benefit
\item lots of incentives to protect it, low incentives to kill it
\end{enumerate}
\end{multicols}

\begin{multicols}{3}
[
\section{Profix Maximization}
Pricing decisions - for firms with market power: Basic principal is $Marginal_{revenue} = Marginal_{cost}$
]

The price I (a monopoly) should set is \\
$P(Q) = Choke_{Price} - 2Q$

Thus, total revenue is $CP \cdot Q - 2Q^2$ \\
Remember, that marginal revenue is $B-A$ where A is \textit{dilution} and B is the actual revenue gained from selling one more unit. \\

MR is the slope of $TR$ or $\frac{\partial TR} {\partial Q}$

More generally: 
$$
	P = a-bQ \; implies \; MR = a - 2bQ
$$

Conclusion: the profit-max rule is: \textbf{Sell Q at the point which $MR=MC$} \\
Note: take a look at Session 7 Page 15 \\
Extensions of this conclusion:
\begin{enumerate}
\item if $MR > MC$ we haven't gone far enough in production
\item if $MR < MC$ we have gone too far so we need to decrease Q
\item at $MR = MC$ we are at the point of max profit
\end{enumerate}

Remember: \textbf{Pricing to cover ATC is an example of the sunk cost fallacy}

An Equivalent to MR=MC: \\
$$
	Lerner Index == \frac{P-c}{P} = \frac{-1}{E_d} 
$$ Or Percentage mark-up is equal to the negative inverse of elasticity of demand \\

\end{multicols}

\begin{multicols}{3}
[
\section{Oligopoly}
Monopoly and duopoly markets
]
Welfare effects of Monopoly:
\begin{enumerate}
\item Monopolist choose to produce where $MR=MC$
\item monopoly leads to higher prices and lower consumption
\item this decreases consumer surplus
\item increases in profits do not compensate for the decrease in consumer surplace
\item this creates dead-weight-loss!
\end{enumerate}

Let's customize prices:
\begin{itemize}
\item strong micro-market: higher WTP and less price sensitive
\item weak micro-market: lower WTP and more price sensitive
\item TOTAL market: sum of the micro-market demand curves 
\end{itemize}
Conclusion: profit contribution of 1 and 2 is greater than simply 3 (single sweet spot price) \\
\hrule

For uniform price:
\textbf{Remember!} Invert, multiply, and differentiate, then set $MR=MC$ to find optimal Q \\

\hrule 

For optimal customized pricing: \textbf{Remember!} set $MR=MC$ in each micro-market \\

To see this: set the uniform price against two \textbf{different} elasticities in each micro-market and compute marginal revenue. As long as MR differs across markets, we can increase profits by customizing prices. 
\hrule 


\end{multicols}

\begin{multicols}{3}
Cournot: The market price depends on the supply decisions of each firm \\
\textbf{Key point:} varying how much quantities supplied, firm can materially influence the market price
Rection functions under Cournot:
\begin{enumerate}
\item start with the inverse demand curve
\item Rearrange  the terms such that $P = [a-bQ_2] - cQ_1$
\item Solve for MR using partial to Q1 such that $MR = [a-bQ_2] - 2cQ_1$
\item solve for $Q_1$

\end{enumerate}

A key takeaway for Cournot: \textbf{Cournot Equ is not efficient because its prices are lower than Monopoly prices, but higher than Perfect Competition Prices}
\end{multicols}
\hrule 

\begin{multicols}{3}
Bertrand: involves each firm charing a penny above its common MC
\begin{enumerate}
\item Expectations shape incentives and drive outcomes
\item If I lower my price by a pennty while my competition does not, I'll gain the increase in quantity solde, so I will do it
\end{enumerate}

Differentiation 
\begin{enumerate}
\item Vertical differentation - firm can charge a higher price for any given price choice of its competitors 
\item horizontal differentation - weak horizontal suggests lots of switchers in total demand while strong horizontal differentiation suggests mostly loyalists account for total demand 
\end{enumerate}

A key takeaway for Bertrand: \textbf{the price competition of two firms lead to efficient outcomes - it's a form of pure strategy under Nash Equ}
\end{multicols}

\begin{multicols}{3}
[
\section{Game Theory}
Simultaneous and Sequential - games model strategic interaction
]

Domination strategy - \textbf{if a player has one action that outperforms all his other actions no matter what the other players do}
Rational players will always use dominant strategies: for example\\
\begin{align*}
&	\Pi(Ad, Ad) > \Pi(NoAd, Ad) \\
&	\Pi(Ad, NoAd) > \Pi(NoAd, NoAd)
\end{align*}

Nash Equ: \textbf{an action profile such that each player's action maximizes his payoff given the actions of other players}

\end{multicols}


\end{document}